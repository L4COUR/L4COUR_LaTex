\documentclass[journal]{IEEEtran}
\ifCLASSINFOpdf
\else
\fi
\hyphenation{op-tical net-works semi-conduc-tor}
\usepackage{cite}
\begin{document}

\title{Showroom: Project Pinocchio}

\author{Frederik~la Cour,~\IEEEmembership{stud.it Digital Design,~Aarhus~University,}
\date{18. Oktober~2018}
\thanks{}
\thanks{}}

\markboth{Student Nr.~201707873,~18. Oktober~2018, anslag: 5107/4800}%
%
{La Cour \MakeLowercase{\textit{}}: Showroom: Project Pinocchio}

\maketitle{Abstract}
\begin{abstract}
The abstract goes here.
\end{abstract}


\begin{IEEEkeywords}
IEEE, IEEEtran, journal, \LaTeX, paper, template.
\end{IEEEkeywords}






% For peer review papers, you can put extra information on the cover
% page as needed:
% \ifCLASSOPTIONpeerreview
% \begin{center} \bfseries EDICS Category: 3-BBND \end{center}
% \fi
%
% For peerreview papers, this IEEEtran command inserts a page break and
% creates the second title. It will be ignored for other modes.
\IEEEpeerreviewmaketitle



\maketitle{Introduction}
\IEEEPARstart{I} will in this paper reflect upon a design project called ‘Pinocchio’ which I co-created alongside my peers. The paper is split into two sections; The first section will be a very brief explanation of the design, in terms of what it is and how it functions. The second section is concerned with discussing how ‘Pinocchio’ fits into a speculative design practice.There are a lot of areas within this topic to reflect on, far more than I would be able to cover in two pages, I have therefor chosen to narrow it down to reflecting on, why the design is called ‘Pinocchio’?

\section{A brief explanation of the design:}
Pinocchio is a piece of wearable technology that consists of three main parts; a set of gloves with build in inductor components, a shoulder-patch with a pair of red and green LED’s, and finally a pair of headphones. All of these parts are connected with a bunch of different cables for audio signals or controlling the LED brightness.

This design explores and utilizes the physical phenomenon of electromagnetic fields(EF), via a small component called an inductor, which is capable of detecting EF as electrical impulses. These impulses can be transcoded into amplified audible frequencies, that will vary depending on the particular EF emitted from an electrical device. When interacting with any device while wearing ‘Pinocchio’ it enables the user to listen to the electrical signals. Not only is it occupying the visual, but also the auditive sensory experience of the interaction, thus resulting in a more intensified and immersed experience.

This intensified interaction also results in making the user unavailable of attaining any form of social interaction, meaning that you can’t be looking on your phone and talking with others simultaneously, any conversational attempts will drown in a wall of EF generated noise sent directly into your ears through the headphones. We found ourselves heavily inspired by Goffman’s term ‘involvement shield’[1, p. 39], which is used to describe a situation where a human engage in the development of a barrier between the human and the social gathering in a public space.

Our design is situated in a future in which this barrier is exaggerated. Imagining a society in which everyone is wearing this contraption, resulting in a transhumanistic sensory increased awareness of technology, that will cause social interaction to become equal to that of interacting with our technology.

\section{In what way does this particular design present itself as speculative?}
By imagining this dystopian future, we want to address our rigorous behavior with technology in a social context, and use the design as a medium to speculate with, rather than attempt to reach this future[2]. Using this dystopian future as a means of discussion, serves not only to contemplate on a posthuman state where our interaction with technology is equal to that of social interaction, but also how we currently, at this point in time, should reflect more upon our dependency of advanced technology. It is important to note that speculative design is not the only future oriented practice. However, what distinguishes speculative design practice from the others are the fact that it is;

’ (…) very interested in positioning design speculation in relation to futurology, speculative culture including literature and cinema, fine art, and radical social science concerned with changing reality rather than simply describing it or maintaining it’ [2, p. 3].

If we look at our design project in relation to this statement, then one of the factors is that if our design speculation is based on speculative literature, or in other words fiction which contains attributes as the supernatural, fantastical, or futuristic elements, then we are operating within a speculative design practice. With that in mind I would like to elaborate on why our design is called ‘Pinocchio’ in the first place.

The inspiration to the name came from a song called ‘I’ve Got no Strings’ from the 1940 Walt Disney animated version of the children’s novel ‘The Adventures of Pinocchio’. Pinocchio is a supernatural wooden doll that that moves without any strings attached to his limps. The connection between this character and our design is that Pinocchio believes that he is free, because he doesn’t need strings in order to move. In the same way you could argue that we humans also perceive ourselves as being free.

We may oppose this thought through speculating with our design. Aren’t we bound to our technology both in a metaphysical sense but also in a literal way, because of how our design portrays itself as being this contraption of modules connected with wires, that have a close resemblance to strings when worn, resulting in not only making us look like cyborgs from a cyberpunk sci-fi novel, but also leads us to further reflection on human behavior within a techno deterministic understanding.

\subsection{Subsection Heading Here}
Subsection text here.

% needed in second column of first page if using \IEEEpubid
%\IEEEpubidadjcol

\subsubsection{Subsubsection Heading Here}
Subsubsection text here.


% An example of a floating figure using the graphicx package.
% Note that \label must occur AFTER (or within) \caption.
% For figures, \caption should occur after the \includegraphics.
% Note that IEEEtran v1.7 and later has special internal code that
% is designed to preserve the operation of \label within \caption
% even when the captionsoff option is in effect. However, because
% of issues like this, it may be the safest practice to put all your
% \label just after \caption rather than within \caption{}.
%
% Reminder: the "draftcls" or "draftclsnofoot", not "draft", class
% option should be used if it is desired that the figures are to be
% displayed while in draft mode.
%
%\begin{figure}[!t]
%\centering
%\includegraphics[width=2.5in]{myfigure}
% where an .eps filename suffix will be assumed under latex,
% and a .pdf suffix will be assumed for pdflatex; or what has been declared
% via \DeclareGraphicsExtensions.
%\caption{Simulation results for the network.}
%\label{fig_sim}
%\end{figure}

% Note that the IEEE typically puts floats only at the top, even when this
% results in a large percentage of a column being occupied by floats.


% An example of a double column floating figure using two subfigures.
% (The subfig.sty package must be loaded for this to work.)
% The subfigure \label commands are set within each subfloat command,
% and the \label for the overall figure must come after \caption.
% \hfil is used as a separator to get equal spacing.
% Watch out that the combined width of all the subfigures on a
% line do not exceed the text width or a line break will occur.
%
%\begin{figure*}[!t]
%\centering
%\subfloat[Case I]{\includegraphics[width=2.5in]{box}%
%\label{fig_first_case}}
%\hfil
%\subfloat[Case II]{\includegraphics[width=2.5in]{box}%
%\label{fig_second_case}}
%\caption{Simulation results for the network.}
%\label{fig_sim}
%\end{figure*}
%
% Note that often IEEE papers with subfigures do not employ subfigure
% captions (using the optional argument to \subfloat[]), but instead will
% reference/describe all of them (a), (b), etc., within the main caption.
% Be aware that for subfig.sty to generate the (a), (b), etc., subfigure
% labels, the optional argument to \subfloat must be present. If a
% subcaption is not desired, just leave its contents blank,
% e.g., \subfloat[].


% An example of a floating table. Note that, for IEEE style tables, the
% \caption command should come BEFORE the table and, given that table
% captions serve much like titles, are usually capitalized except for words
% such as a, an, and, as, at, but, by, for, in, nor, of, on, or, the, to
% and up, which are usually not capitalized unless they are the first or
% last word of the caption. Table text will default to \footnotesize as
% the IEEE normally uses this smaller font for tables.
% The \label must come after \caption as always.
%
%\begin{table}[!t]
%% increase table row spacing, adjust to taste
%\renewcommand{\arraystretch}{1.3}
% if using array.sty, it might be a good idea to tweak the value of
% \extrarowheight as needed to properly center the text within the cells
%\caption{An Example of a Table}
%\label{table_example}
%\centering
%% Some packages, such as MDW tools, offer better commands for making tables
%% than the plain LaTeX2e tabular which is used here.
%\begin{tabular}{|c||c|}
%\hline
%One & Two\\
%\hline
%Three & Four\\
%\hline
%\end{tabular}
%\end{table}


% Note that the IEEE does not put floats in the very first column
% - or typically anywhere on the first page for that matter. Also,
% in-text middle ("here") positioning is typically not used, but it
% is allowed and encouraged for Computer Society conferences (but
% not Computer Society journals). Most IEEE journals/conferences use
% top floats exclusively.
% Note that, LaTeX2e, unlike IEEE journals/conferences, places
% footnotes above bottom floats. This can be corrected via the
% \fnbelowfloat command of the stfloats package.




\section{Conclusion}
The conclusion goes here.





% if have a single appendix:
%\appendix[Proof of the Zonklar Equations]
% or
%\appendix  % for no appendix heading
% do not use \section anymore after \appendix, only \section*
% is possibly needed

% use appendices with more than one appendix
% then use \section to start each appendix
% you must declare a \section before using any
% \subsection or using \label (\appendices by itself
% starts a section numbered zero.)
%


\appendices
\section{Proof of the First Zonklar Equation}
Appendix one text goes here.

% you can choose not to have a title for an appendix
% if you want by leaving the argument blank
\section{}
Appendix two text goes here.


% use section* for acknowledgment
\section*{Acknowledgment}


The authors would like to thank...


% Can use something like this to put references on a page
% by themselves when using endfloat and the captionsoff option.
\ifCLASSOPTIONcaptionsoff
  \newpage
\fi



% trigger a \newpage just before the given reference
% number - used to balance the columns on the last page
% adjust value as needed - may need to be readjusted if
% the document is modified later
%\IEEEtriggeratref{8}
% The "triggered" command can be changed if desired:
%\IEEEtriggercmd{\enlargethispage{-5in}}

% references section

% can use a bibliography generated by BibTeX as a .bbl file
% BibTeX documentation can be easily obtained at:
% http://mirror.ctan.org/biblio/bibtex/contrib/doc/
% The IEEEtran BibTeX style support page is at:
% http://www.michaelshell.org/tex/ieeetran/bibtex/
%\bibliographystyle{IEEEtran}
% argument is your BibTeX string definitions and bibliography database(s)
%\bibliography{IEEEabrv,../bib/paper}
%
% <OR> manually copy in the resultant .bbl file
% set second argument of \begin to the number of references
% (used to reserve space for the reference number labels box)
\begin{thebibliography}{1}

\bibitem{IEEEhowto:kopka}
H.~Kopka and P.~W. Daly, \emph{A Guide to \LaTeX}, 3rd~ed.\hskip 1em plus
  0.5em minus 0.4em\relax Harlow, England: Addison-Wesley, 1999.

\end{thebibliography}

% biography section
%
% If you have an EPS/PDF photo (graphicx package needed) extra braces are
% needed around the contents of the optional argument to biography to prevent
% the LaTeX parser from getting confused when it sees the complicated
% \includegraphics command within an optional argument. (You could create
% your own custom macro containing the \includegraphics command to make things
% simpler here.)
%\begin{IEEEbiography}[{\includegraphics[width=1in,height=1.25in,clip,keepaspectratio]{mshell}}]{Michael Shell}
% or if you just want to reserve a space for a photo:

% insert where needed to balance the two columns on the last page with
% biographies
%\newpage

% You can push biographies down or up by placing
% a \vfill before or after them. The appropriate
% use of \vfill depends on what kind of text is
% on the last page and whether or not the columns
% are being equalized.

%\vfill

% Can be used to pull up biographies so that the bottom of the last one
% is flush with the other column.
%\enlargethispage{-5in}



% that's all folks
\end{document}
